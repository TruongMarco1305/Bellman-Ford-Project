\documentclass[a4paper]{article}
\usepackage{import}
%\usepackage[english,vietnam]{babel}
\usepackage[utf8]{inputenc}
%\usepackage[utf8]{inputenc}
%\usepackage[francais]{babel}
\usepackage{a4wide,amssymb,epsfig,latexsym,array,hhline,fancyhdr}
\usepackage[normalem]{ulem}
%\usepackage{soul}
\usepackage{listings}
\usepackage{colortbl}
\usepackage[makeroom]{cancel}
\usepackage{amsmath}
\usepackage{amsthm}
\usepackage{multicol,longtable,amscd}
\usepackage{diagbox}%Make diagonal lines in tables
\usepackage{booktabs}
\usepackage{alltt}
\usepackage[framemethod=tikz]{mdframed}% For highlighting paragraph backgrounds
\usepackage{caption,subcaption}

\usepackage{lastpage}
\usepackage[lined,boxed,commentsnumbered]{algorithm2e}
\usepackage{enumerate}
\usepackage{color}
\usepackage{graphicx}							% Standard graphics package
\usepackage{array}
\usepackage{tabularx, caption}
\usepackage{multirow}
\usepackage{multicol}
\usepackage{rotating}
\usepackage{graphics}
\usepackage{geometry}
\usepackage{setspace}
\usepackage{epsfig}
\usepackage{minted}
\usepackage{xcolor} % to access the named colour LightGray
\definecolor{LightGray}{gray}{0.9}
\usemintedstyle{emacs}
\usepackage{tikz}
\usetikzlibrary{arrows,snakes,backgrounds}
\usepackage[unicode]{hyperref}
\hypersetup{
    urlcolor=blue,
    linkcolor=black,
    citecolor=black,
    colorlinks=true,
    pdfpagemode=FullScreen,
    pdftitle={The Travelling Salesman Problem},
} 
%\usepackage{pstcol} 								% PSTricks with the standard color package
%\usepackage{background}
%\backgroundsetup{contents=\includegraphics{Images/hcmut.png}, scale=0.5, opacity=0.25, angle = 0}
\usepackage[normalem]{ulem}

\def\thesislayout{	% A4: 210 × 297
	\geometry{
		a4paper,
		total={160mm,240mm},  % fix over page
		left=30mm,
		top=30mm,
	}
}
\thesislayout

\usepackage{fancyhdr}
\setlength{\headheight}{40pt}
\pagestyle{fancy}
\fancyhead{} % clear all header fields
\fancyhead[L]{
 \begin{tabular}{rl}
    \begin{picture}(25,15)(0,0)
    \put(0,-8){\includegraphics[width=8mm, height=8mm]{Images/hcmut.png}}
    %\put(0,-8){\epsfig{width=10mm,figure=hcmut.eps}}
   \end{picture}&
	%\includegraphics[width=8mm, height=8mm]{hcmut.png} & %
	\begin{tabular}{l}
		\textbf{\textcolor{blue}{\bf \ttfamily Ho Chi Minh City University of Technology}}\\
		\textbf{\textcolor{blue}{\bf \ttfamily Faculty of Computer Science and Engineering}}
	\end{tabular} 	
 \end{tabular}
}
\fancyhead[R]{
	\begin{tabular}{l}
		\tiny \bf \\
		\tiny \bf 
	\end{tabular}  }
\fancyfoot{} % clear all footer fields
\fancyfoot[L]{\scriptsize \ttfamily Truong Gia Ky Nam - 2352787}
\fancyfoot[R]{\scriptsize \ttfamily Page {\thepage}/\pageref{LastPage}}
\renewcommand{\headrulewidth}{0.3pt}
\renewcommand{\footrulewidth}{0.3pt}


%%%
\setcounter{secnumdepth}{4}
\setcounter{tocdepth}{3}
\makeatletter
\newcounter {subsubsubsection}[subsubsection]
\renewcommand\thesubsubsubsection{\thesubsubsection .\@alph\c@subsubsubsection}
\newcommand\subsubsubsection{\@startsection{subsubsubsection}{4}{\z@}%
                                     {-3.25ex\@plus -1ex \@minus -.2ex}%
                                     {1.5ex \@plus .2ex}%
                                     {\normalfont\normalsize\bfseries}}
\newcommand*\l@subsubsubsection{\@dottedtocline{3}{10.0em}{4.1em}}
\newcommand*{\subsubsubsectionmark}[1]{}
\makeatother

\sloppy
\captionsetup[figure]{labelfont={small,bf},textfont={small,it},belowskip=-1pt,aboveskip=-9pt}
%space remove between caption, figure, and text
\captionsetup[table]{labelfont={small,bf},textfont={small,it},belowskip=-1pt,aboveskip=7pt}
%space remove between caption, table, and text

%\floatplacement{figure}{H}%forced here float placement automatically for figures
%\floatplacement{table}{H}%forced here float placement automatically for table
%the following settings (11 lines) are to remove white space before or after the figures and tables
%\setcounter{topnumber}{2}
%\setcounter{bottomnumber}{2}
%\setcounter{totalnumber}{4}
%\renewcommand{\topfraction}{0.85}
%\renewcommand{\bottomfraction}{0.85}
%\renewcommand{\textfraction}{0.15}
%\renewcommand{\floatpagefraction}{0.8}
%\renewcommand{\textfraction}{0.1}
\setlength{\floatsep}{5pt plus 2pt minus 2pt}
\setlength{\textfloatsep}{5pt plus 2pt minus 2pt}
\setlength{\intextsep}{10pt plus 2pt minus 2pt}

\thesislayout



\begin{document}
\begin{titlepage}
\begin{center}
\textbf{\Large VIETNAM NATIONAL UNIVERSITY HO CHI MINH CITY} \\

\vspace{7pt}
\textbf{\Large HO CHI MINH CITY UNIVERSITY OF TECHNOLOGY} \\

\vspace{7pt}
\textbf{\Large FACULTY OF COMPUTER SCIENCE AND ENGINEERING}
\end{center}

\vspace{1cm}

\begin{figure}[h!]
\begin{center}
\includegraphics[width=3cm]{Images/hcmut.png}
\end{center}
\end{figure}

\vspace{1cm}


\begin{center}
\begin{tabular}{ccc}
	\multicolumn{3}{l}{\textbf{{\Large \textcolor{blue}{DISCRETE STRUCTURE}}}}\\
	~~\\
	\arrayrulecolor{blue}\hline
	\\
	\multicolumn{3}{l}{\textbf{{\Large \textcolor{blue}{Assignment 1} }}}\\
	\\
	
	\multicolumn{3}{c}{\textbf{{\huge \textcolor{blue}{Using Dynamic Programming}}}}\\
	\\
    
    \multicolumn{3}{c}{\textbf{{\huge \textcolor{blue}{To Solve The Travelling Saleman Problem}}}}\\
	\\
	\arrayrulecolor{blue}\hline \\ \\

    \multicolumn{1}{r}{\textbf{\Large Instructors:}} & \multicolumn{2}{l}{\Large Nguyen Van Minh Man, \textit{Mahidol University}} \\ \\
    \multicolumn{1}{r}{} & \multicolumn{2}{l}{\Large Tran Tuan Anh, \textit{CSE-HCMUT}} \\ \\

    \\ \\

    \multicolumn{1}{r}{\Large \textbf{Author:}} & \multicolumn{2}{l}{\Large Truong Gia Ky Nam} \\ \\

    \multicolumn{1}{r}{\Large \textbf{ID:}} & \multicolumn{2}{l}{\Large 2352787} \\ \\

    \multicolumn{1}{r}{\Large \textbf{Email:}} & \multicolumn{2}{l}{\Large nam.truonggiaky@hcmut.edu.vn} \\ \\
\end{tabular}
\end{center}

\vspace{4cm}

\begin{center}
{\textbf{\Large Ho Chi Minh City, May 2024}}
\end{center}
\end{titlepage}

\thispagestyle{empty}
\setcounter{page}{-1}
\newpage
\tableofcontents
\newpage

\thispagestyle{empty}
\newpage
\begin{abstract}
\noindent This document was made during the class CC03 from course Discrete Structure semester 232 with the instruction from Dr. Nguyen Van Minh Man and Dr. Tran Tuan Anh. This document serves as the assignment report for the Assignment 1 that Dr. Minh Man and Dr. Tuan Anh gave us during the course. By gather information in searching information on the Internet, our group is able to gather required information about the topic that is given and summarized in this report.
\end{abstract}
\newpage

%\thispagestyle{empty}
\section{Introduction}

In this report, I will show how we can use the dynamic programming technique to solve the Traveling Saleman Problem with clearly explaination about the approaching method and the purpose of each functions of the source code. 

\section{The Traveling Salemen Problem}
\subsection{History}

\subsection{Approaching}

\subsection{tspprogram}

\begin{minted}[
	frame=lines,
	framesep=2mm,
	baselinestretch=1.2,
	fontsize=\footnotesize,
	linenos
	]{cpp}
void tspproblem(){
    int cityMap[numberOfVertices][numberOfVertices];
    for (int i = 0; i < numberOfVertices; i++){
        for (int j = 0; j < numberOfVertices; j++){
            if (i == j){
                cityMap[i][j] = 0;
            }else{
                cout << "Input the distance from city " << char(i + 65) << " to city " << char(j + 65) << ": ";
                cin >> cityMap[i][j];
            }
        } 
    }

    char startCity;
    cout << "Input the start city: ";
    cin >> startCity;

    Traveling(cityMap,startCity);
}
\end{minted}
\subsection{Travling function}
\begin{minted}[
	frame=lines,
	framesep=2mm,
	baselinestretch=1.2,
	fontsize=\footnotesize,
	linenos
	]{cpp}
void Traveling(int cityMap[][numberOfVertices],char startVertex){
    vector<vector<int>> memo(numberOfVertices,vector<int>(1 << numberOfVertices,0));
    int source = int(startVertex - 'A');

    setUp(cityMap,memo,source,numberOfVertices);
    solve(cityMap,memo,source,numberOfVertices);
    
    cout << findOptimalTour(cityMap,memo,source,numberOfVertices);
}
\end{minted}

\subsection{}

\subsection{}

\subsection{}

\subsection{Dynamic Programming}

\section{Conclusion}
BLA BLA BLA

\pagebreak

\begin{minted}[
frame=lines,
framesep=2mm,
baselinestretch=1.2,
fontsize=\footnotesize,
linenos
]{cpp}
#include <iostream>

using namespace std;

int main(){
    //Use this to print Hello World
    cout << "Hello World";
    return 0;
}
\end{minted}
\begin{thebibliography}{80}

\bibitem{CVX}
CVX Introduction
``\textbf{link: http://cvxr.com/cvx/doc/intro.html/}'',
\textit{What is CVX}.

\end{thebibliography}
\end{document}

